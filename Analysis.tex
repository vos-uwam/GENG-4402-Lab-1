\section{Results}
  \subsection{Frequency Response Experiment}
  Steady state gain was calculated by taking the ratio of load shaft speed to constant input, then converted to a decible amplitude using
  $$ G_{dB} = 20 \cdot log_{10}(G)$$ from the workbook \cite{workbook}. For each frequency step up to 8 Hz, the gain was calculated and
  tabulated in table \ref{tab1.1}.\\
  The results were plotted in the bode plot, figure %\ref{figure}%. Ginput(1) was used to find the -3dB cutoff frequency, $$f_{3dB}$$.
  The time constant was calculated using $$\tau = \frac{1}{2 \cdot \pi \f_{3dB}}. The time constant and steady state gain were tabulated
  in table \ref{tab1.2}.\\
  
  \subsection{Bump Test Experiment}
  The open loop gain was calculated by taking the ratio of step change to the difference between steady state speed and initial speed.
  That is, the height of the input voltage trace in figure %\ref{figure}% compared to the height of the steady state motor speed on the same figure.
  Ginput(1) was used to obtain the maximum and minimum values for calculation.\\
  The expression $$ y(t_{1}) = 0.632 \cdot y_{ss} + y_{0}$$ from the workbook gives the time $$ t_{1} $$ required to determine $$\tau$$ \cite{workbook}.
  $$y(t_{1})$$ was found in the Matlab command line, then $$t_{1}$$ was found using ginput(1). The time constant was calculated using 
  $$\tau = t_{1} - t_{0}$$, then tabulated along with the gain into table \ref{tab1.2}.
